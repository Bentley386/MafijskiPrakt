\documentclass{article}

\title{Matemati{\v c}no-fizikalni praktikum: Diferenčne metode za PDE}
\author{Andrej Kolar-Po{\v z}un}

\usepackage[utf8]{inputenc}
\usepackage{amsmath}
\usepackage{graphicx}
\usepackage{subcaption}
\usepackage{caption}
\usepackage{float}
\usepackage{physics}

\errorcontextlines 10000
\begin{document}
\pagenumbering{gobble}
\maketitle
\newpage
\pagenumbering{arabic}

\section{Uvod}

V tej nalogi bomo reševali Schrödingerjevo enačbo:
\begin{align*}
&\Big(i\hbar \frac{\partial}{\partial t} - H \Big) \psi(x,t) = 0 \\
&H = -\frac{\hbar ^2}{2m} \frac{\partial ^2}{\partial x^2} + V(x) 
\end{align*}

S pomočjo operatorja časovnega razvoja lahko pridemo do tele aproksimacije:
Časovni razvoj spremljamo ob časih $t_n = n \Delta t$
\begin{equation*}
\psi(x,t+\Delta t) \approx \frac{1-\frac{1}{2}i H \Delta t}{1+\frac{1}{2}i H \Delta t}\psi (x,t)
\end{equation*}

Po kraju pa lahko območje, ki nas zanima diskretiziramo kot $x_j = a+j \Delta x$ $\Delta x = (b-a)/(N-1)$ in krajevni odvod
aproksimiramo kot diferenco:
\begin{equation*}
\psi ''(x) = \frac{\psi_{j+1}^n - 2\psi_j^n + \psi_{j-1} ^n}{\Delta x^2}
\end{equation*}
Kjer smo uvedli oznako $\psi(x_j,t_n) = \psi_j^n$

Ko ti aproksimaciji vstavimo v enačbo dobimo sistem enačb katerega lahko zapišemo v matrični
obliki kot $A\psi^{n+1} = A^* \psi^n$, kjer je $\psi ^n = (\psi_0^n,\psi_1^n,...,\psi_{N-1}^n)$





\end{document}
