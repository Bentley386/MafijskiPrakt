\documentclass{article}

\title{Mafija praktikum: Airyjevi funkciji}
\author{Andrej Kolar-Po{\v z}un}

\usepackage[utf8]{inputenc}
\usepackage{amsmath}
\usepackage{graphicx}
\usepackage{caption}
\usepackage{float}

\begin{document}
\pagenumbering{gobble}
\maketitle
\newpage
\pagenumbering{arabic}

\section{Uvod}
V temle dokumentu se bom lotil reševanja prve naloge matematično-fizikalnega praktikuma. Navodila za nalogo in potrebni podatki so v isti mapi pod imenom
"mafijaprakt1.pdf".
Naloge se bom lotil s Pythonom 3.5 (SciPy knjižnice).

\section{Računanje potrebnih vrst}

Vidimo, da lahko funkciji Ai(x) ter Bi(x) izračunamo preko raznih podanih vrst. Dobro bi bilo torej začeti s tem, da
ustvarimo funkcije, ki bodo lahko seštele poljubno(končno) število členov te vrste.

Poglejmo si najprej tole vrsto:
\begin{equation*}
f(x) = \sum_{k=0}^{\infty} \left(\frac{1}{3}\right)_k \frac{3^k x^{3k}}{(3k)!}
\end{equation*}

Vemo, da bo za natančen rezultat treba sešteti zelo veliko členov te vrste. V vrsti nastopa veliko hitro rastočih funkcij: potenčna, 
fakulteta in gama funkciji v definiciji Pochammerjevega simbola. 
Takim ogromnim številkam se raje izogibamo in poskusimo vrsto izračunati na pametnejši način. Zapišimo nekaj prvih členov vrste:


\begin{align*}
a_0 &= 1 \\
a_1 &= a_0 * \frac{3x^3}{3*2*1}*\frac{1}{3} \\
a_2 & = a_1 * \frac{3x^3}{6*5*4}*\frac{4}{3}
\end{align*}

Kjer smo uporabili identiteto $\Gamma(z+1) = z*\Gamma(z)$.
Sedaj lahko z malo premisleka uganemo rekurzivno formulo za poljuben člen $a_k$, kjer k ni nič.

\begin{equation*}
a_k = a_{k-1} * \frac{3x^3}{3k*(3k-1)*(3k-2)}*(k-\frac{2}{3})
\end{equation*}

Na isti način dobimo tudi rekurzivne zveze za člene ostalih vrst, ki jih moramo izračunati.

\subsection{Vrsta g(x)}
\begin{align*}
a_0 &= x \\
a_k &= a_{k-1} * \frac{3x^3}{(3k-1)3k(3k+1)}*(k-\frac{1}{3})
\end{align*}

\subsection{Vrsta L(z)}
\begin{align*}
a_0 &= 1 \\
a_k &= a_{k-1} * \frac{(3s-\frac{5}{2})(3s-\frac{3}{2})(3s-\frac{1}{2})}{54zs(s-\frac{1}{2})}
\end{align*}

\subsection{Vrsta P(z)}
\begin{align*}
a_0 &= 1 \\
a_k &= -a_{k-1} *  \frac{(6s-\frac{11}{2})(6s-\frac{9}{2})(6s-\frac{7}{2})....(6s-\frac{1}{2})}{54^2z^22s(2s-1)(2s-\frac{3}{2})(2s-\frac{1}{2})}
\end{align*}

\subsection{Vrsta Q(z)}
\begin{align*}
a_0 &= \frac{5}{72z}\\
a_k &= -a_{k-1}* \frac{(6s-\frac{5}{2})(6s-\frac{3}{2}).....(6s+\frac{5}{2})}{54^2z^22s(2s+1)(2s-\frac{1}{2})(2s+\frac{1}{2})}
\end{align*}



Največje vprašanje je, koliko členov vrste moramo sešteti, da dobimo zadovoljiv približek. 
Za začetek sem naredil funkcije, ki vrste seštevajo dokler členi niso manjši od določene številke/dokler seštetih členov ni več od določene številke. To bom poskusil ugotoviti z malo probavanja.

\newpage

\section{Aproksimacije za funkcijo Ai}

\subsection{Vrsta za majhne x}

Izbral sem si(kar tako, da vidim kaj se zgodi), da bom nehal seštevati, ko členi postanejo manjši od $10^{-12}$ in to uporabil, da bi dobil vrednost Ai(x) v točki x=5. Sesteti sem moral le okoli 20 clenov iz vsake vrste, da bi se priblizal pravemu rezultatu na $10^-13$ natančno.

Sedaj hočem bolj podbrobno raziskati do kje približno tale vrsta dobro in hitro deluje in kdaj naj končam seštevat.

Preizkušal sem malo različne možnosti in primerjal z pravim rezultatom:
\newpage

\begin{figure}[H]

\includegraphics{airy/majhenfiksenn.png}
\caption*{Tukaj sem vedno šeštel določeno število členov n}
\end{figure}
\begin{figure}[H]
\includegraphics{airy/majhenfiksennblizje.png}
\caption*{Povečan del grafa, kjer se začne močno odstopanje}
\end{figure}
\newpage
\begin{figure}[H]
\includegraphics{airy/majhennfiksene.png}
\caption*{Tukaj sem šešteval dokler ni naslednji člen manjši od določene vrednosti}
\end{figure}
\begin{figure}[H]
\includegraphics{airy/majhenrazlika.png}
\caption*{Odstopanje od prave vrednosti v odvisnosti od x}
\end{figure}
\newpage

Vidimo torej, da ni neke opazne razlike, če računamo določeno število členov, ali se ustavimo ko so ti členi dovolj majhni.
Izgleda, da bom dobil spodobno majhno odstopanje če vrsto za majhne x uporabim le za približno $|x| < 5$
Člene pa bom nehal šeštevati, ko bo velikost člena padla pod $10^{-12}$, saj sem opazil, da do tega hitro pride($n << 1000$) in je torej tudi hitrejše.

\subsection{Vrsta za velike x}
 
Najprej sem optimistično poskusil vrsto nehati seštevati po istem kriteriju kot za male x. Hitro se je opazilo, da bo tukaj drugace, saj členi sploh niso nikoli padli pod $10^{-12}$. Poskusil sem zmanjšati pogoj na $10^{-10}$ a ni pomagalo. Sedaj sem se nehal ozirati na velikost členov in seštel kar fiksno število členov, da sploh lahko kaj narišem in vidim, če je vsaj nekje približno prav.
Seštevanje fiksnega števila je še slabše, vrže zelo velike vrednosti in pogosto NaN(not a number).
Še enkrat sem poskusil z končanjem seštevanja ko členi padejo pod določeno vrednost. Tokrat sem bil bolj blag in določil mejo na $10^{-7}$:

\begin{figure}[H]
\includegraphics{airy/velikxsto.png}
\caption*{Aproksimacija z modro barvo, prava vrednost z rdečo}
\end{figure}

Izgleda torej, da se kar dobro prilega, takoj bom pogledal še koliko točno in preveril obnašanje še na višjih številkah. Opazil sem, da pri asimptostki vrsti sploh ni toliko vprašanje pri katerih parametrih se bolje prilega, ampak pri katerih parametrih sploh kaj vrže ven. Če neham šeštevati pri še manjši vrednosti se hitro zgodi, da šeštejem že milijon členov ali več. Opazil sem tudi, da pri majhih argumentih(pod približno $|x|<4$) začne zadeva močno divergirati. Členi pridejo do "inf", končen rezultat pa je "NaN" torej je ta vrsta za tako majhne argumente neuporabna.
Pri šeštevanju fiksnih členov za večje argumente tudi hitro pridemo do "NaN". To ni tako presenetljivo, saj vemo, da so asimptotične vrste pogosto divergentne. Verjetno smo pri šeštevanju že prešli pravo vrednost in so členi začeli divjo naraščati.

Prejšnjo metodo, ki je izgledala obetavna sem sedaj preizkusil še na širšem območju. Rezultat:

\begin{figure}[H]
\includegraphics{airy/velikxtisoc.png}
\caption*{Aproksimacija z modro barvo, prava vrednost z rdečo}
\end{figure}

Lepo se vidi, da je pri približno $x=100$ nekaj narobe. Morda zato, ker so tam vrednosti Airyjeve funkcije veliko manjše od mojega pogoja za končanje seštevanja vrste.

Poskusil bom to rešiti na čisto drugačen način. Asimptosko vrsto bom šešteval dokler je trenutni člen po absolutni vrednosti manjši od prejšnjega. Da vidimo ali bo to delovalo..

\begin{figure}[H]
\includegraphics{airy/problem.png}
\caption*{Aproksimacija z modro barvo, prava vrednost z rdečo}
\end{figure}

Stvar torej težave ni rešila. Vseeno bom ostal pri tej metodi, saj se je zdela hitra in bolj smiselno se zdi, da šeštevanje členov ustavimo, ko začnejo naraščati.
Sedaj menim, da te težave mogoče sploh ni treba popraviti. Vrednosti funkcije v tistem območju so izjemno majhne reda $10^{-300}$. Sicer je napaka kar velika, vendar dvomim, da bi takšne točne vrednosti sploh kdaj rabili. V praksi se verjetno temu reče kar 0.

Se je pa zgodilo nekaj zanimivega. Zdaj ko šeštevanje končam po novem načinu, ne prihaja več do NaN napak. Seveda pa je za majhne x rešitev vseeno zelo narobe:

\begin{figure}[H]
\includegraphics{airy/divergenca.png}
\caption*{Aproksimacija z modro barvo, prava vrednost z rdečo}
\end{figure}


Pa poglejmo zdaj za koliko se aproksimacija razlikuje od prave rešitve:

\begin{figure}[H]
\includegraphics{airy/velikixrazlika.png}
\caption*{Razlika med aproksimacijo in pravo vrednostjo}
\end{figure}

Zanimvo. Pri srednje manjših argumentih torej razlika dokaj divje skače. Vseeno ostaja manjša od nekje $10^{-20}$ kar se mi zdi zadovoljiva aproksimacija. Že pri $x=5$ pa je razlika že okoli reda $10^{-12}$, torej ta asimptotska vrsta dokaj dobro deluje že za manjše argumente.
Mimogrede, res, da je pri večjih argumentih vrednost funkcije zelo majhna, a kolikor sem opazil je vrednost napake pri teh istih vrednostih vseeno kakih 10 velikostnih redov manjša.
Poglejmo si kako se razlika obnaša za negativne argumente:

\begin{figure}[H]
\includegraphics{airy/velikixrazlikaminus.png}
\caption*{Razlika med aproksimacijo in pravo vrednostjo}
\end{figure}

Tukaj torej na še večjem območju razlika divja. Sicer pa je tudi funkcija sama v tem območju nemonotona. Se pa zdaj pojavi problem, ker ima funkcija tudi ničle, saj je napaka povsod neničelna torej ne veliko manjša od vrednosti funkcije kot prej. Nimam kakšnih idej za izboljšanje tega, zdi se mi pa, da pri ničlah z metodami aproksimacije tako ali tako sploh ne moremo priti do tega, da bi "absolutne vrednost napake bila nekaj redov manjša od absolutne vrednosti funkcije", napaka pa ostane enako majhna tudi, ko se oddaljimo od ničle, torej se bom s tem kar zadovoljiv.

\subsection{Funkcija Ai(x) za poljuben x}

Zdaj imam torej metodi, ki delujeta za velike in male ikse. Poskusil ju bom združiti, s tem, da bo moj mejnik katero naj uporabim $|x|=7$, saj izgleda  da je tam nekje mejnik kdaj postane ena metoda bolj natančna od druge.

\begin{figure}[H]
\includegraphics{airy/koncen.png}
\caption*{Aproksimacija z modro barvo, prava vrednost z rdečo}
\end{figure}

Za negativne ikse zadeva tako hitro oscilira, da se težko vidi ali se aproksimacija res pokriva z pravo vrednostjo ali ne. Tukaj je torej del negativne osi malce pobližje:

\begin{figure}[H]
\includegraphics{airy/koncenneg.png}
\caption*{Aproksimacija z modro barvo, prava vrednost z rdečo}
\end{figure}

Kaj pa odstopanje na celi osi?

\begin{figure}[H]
\includegraphics{airy/napakacela.png}
\caption*{Razlika med aproksimacijo in pravo vrednostjo}
\end{figure}

\section{Funkcija Bi(x)}

Ker se vse aproksimacije funkcije Bi(x) izražajo z istimi vrstami kot so se Ai(x) sem imel idejo, da bodo mogoče isti pogoji za nehanje šeštevanja vrst delovali
tudi tukaj. In sem poskusil:

\begin{figure}[H]
\includegraphics{bairy/veliko.png}
\caption*{Aproksimacija z modro barvo, prava vrednost z rdečo}
\end{figure}

\begin{figure}[H]
\includegraphics{bairy/minus.png}
\caption*{Aproksimacija z modro barvo, prava vrednost z rdečo}
\end{figure}

\begin{figure}[H]
\includegraphics{bairy/plus.png}
\caption*{Aproksimacija z modro barvo, prava vrednost z rdečo}
\end{figure}

Zgledalo je torej zelo obetavno, a potem sem pogledal kolikšna je napaka aproksimacije:

\begin{figure}[H]
\includegraphics{bairy/napaka.png}
\caption*{Aproksimacija z modro barvo, prava vrednost z rdečo}
\end{figure}

Absolutna napaka torej grozljivo naraste. Na hitro sem poskusil povecati stevilo sestetih clenov, a napaka je ostajala. Potem sem opazil, da
nad približno x=200, zadevo že tako divergira, da v bistvu daja rezultate NaN oz. inf, vseeno so bile razlike ogromne tudi na območju 100-150 kjer
je naprimer funkcija že veliko narastla a ni še "inf/NaN".
Še malo sem gledal vrednosti in končno ugotovil, kje je "težava". Res je da so napake ogromne, vendar je tudi vrednost funkcije na tem območju ogromna. Pravzaprav je vrednost funkcije vseeno kakih 10 velikostnih redov večja od napake, torej to pravzaprav sploh ni tako velika napaka.

\section{Končen pogled}

Narišimo zdaj za konec še obe moji Airyjevi funkciji na isti graf pa še relativno napako:

\begin{figure}[H]
\includegraphics{oboje.png}
\end{figure}

\begin{figure}[H]
\includegraphics{oboje2.png}
\end{figure}

\begin{figure}[H]
\includegraphics{oboje3.png}
\end{figure}

\begin{figure}[H]
\includegraphics{napaki.png}
\caption*{Relativna napaka}
\end{figure}

Napaka aproksimacije Bi(x) je torej zelo zadovoljiva, aproksimacije Ai(x) pa sicer malce manj na tistem območju kjer je nekje meja med aproksimacijo za male in velike argumente.
\newpage
\section{Dodatna naloga}

Sedaj nas zanima prvih 100 ničel Airyjevih funkcij. Prva metoda, ki mi pade na pamet je Newtnova metoda(Ker vgrajena Ai(x) funkcija v Pythonu vrača tudi odvode). Problem je pri izbiri začetnega približka ničle. Najprej si bom malo pobližje ogledal graf Airyjevih funkcij, da vidim koliko narazen so sploh ničle(Čeprav ni nujno, da je ta interval konstanten za vse ničle).

Videl sem, da so prve ničle razmaknjene za ~1-2, pozneje se malo škrči in so razmaknjene za ~0.5, mislim, da bom kar poskusil z Newtnovo metodo iskati ničle s tem da bom začel pri prvi ničli(ki jo bom dobil s začetnim približkom 0) in potem šel negativno recimo po intervalih 0.1 Da vidimo kaj se bo zgodilo..

Poskusil sem torej z Newtnovo metodo, ki pa je zelo pocasi konvergirala(metalo je ven tak error) oziroma je nekako konvergirala proti necemu pozitivnemu(kar ni res saj vemo, da so ničle le na negativni realni osi).

Zdaj sem to(na zelo  slab nacin) popravil. Iskal sem po isti newtnovi metodi, le da sem sproti metal stran ničle ki so večje od nič(ki v resnici sploh niso ničle le močno približevanje ničli pri prvi Airyjevi funkciji) in sproti gledal po celem do sedaj polnem seznamu obstojecih nicel, da ne dobim ponesreci še ene ničle, ki je od prejšnje le 0.001 stran(kar bi v bistvu bila samo ena ničla). Rezultati so sledeči:
\newpage
\subsection{Ničle Ai(x)}
formula:-2.337163322273885, Newton-2.336141997554901

formula:-4.087948607972578, Newton-4.090582746735273

formula:-5.520559812349988, Newton-5.51331712331758

formula:-6.786708089109229, Newton-6.786800985401768

formula:-7.944133587009499, Newton-7.9469220453911005

formula:-9.022650853321732, Newton-9.018933435863268

formula:-10.0401743415537, Newton-10.03328479509225

formula:-11.008524303732038, Newton-11.008377877088847

formula:-11.93601556323586, Newton-11.936273600081728

formula:-12.828776752865613, Newton-12.828772795156679

formula:-13.691489035210658, Newton-13.68982077844435

formula:-14.527829951775306, Newton-14.534897225683839

formula:-15.340755135977986, Newton-15.340688637892825

formula:-16.13268515694576, Newton-16.12621919878334

formula:-16.905633997429938, Newton-16.900342202961234

formula:-17.661300105697052, Newton-17.66168024256383

formula:-18.401132599207113, Newton-18.402288059813305

formula:-19.126380474246947, Newton-19.124309619395888

formula:-19.8381298917215, Newton-19.83071160172986

formula:-20.537332907677563, Newton-20.534546527850225

formula:-21.224829943642096, Newton-21.22705561201599

formula:-21.901367595585125, Newton-21.908856499433707

formula:-22.567612917496497, Newton-22.56760700344918

formula:-23.224165001121676, Newton-23.218746701854528

formula:-23.871564455535914, Newton-23.871392441350583

formula:-24.51030123658968, Newton-24.507369057398147

formula:-25.14082116614896, Newton-25.14127025758591

formula:-25.76353140098275, Newton-25.75955415254264

formula:-26.378805052137228, Newton-26.378328811115445

formula:-26.986985111606362, Newton-26.986940238564124

formula:-27.588387809882438, Newton-27.591757221650262

formula:-28.183305502632635, Newton-28.183300120565658

formula:-28.77200916523743, Newton-28.772475317672534

formula:-29.354750558766284, Newton-29.351788640015258

formula:-29.93176411908655, Newton-29.933821844011714

formula:-30.503268611418495, Newton-30.4968739163465

formula:-31.06946858518375, Newton-31.068937692269795

formula:-31.630555658012657, Newton-31.633527702044873

formula:-32.18670965295205, Newton-32.18866177225079

formula:-32.73809960900026, Newton-32.7380527437487

formula:-33.28488468190139, Newton-33.28522331227882

formula:-33.82721494950864, Newton-33.82695726684468

formula:-34.36523213386365, Newton-34.364603847719586

formula:-34.899070250345304, Newton-34.899373652053406

formula:-35.42885619274788, Newton-35.428636506500006

formula:-35.954710261898626, Newton-35.953332699248485

formula:-36.476746644374806, Newton-36.47686217351186

formula:-36.995073846994494, Newton-36.9950916868033

formula:-37.50979509200501, Newton-37.51143450378521

formula:-38.02100867725524, Newton-38.01974168184952

formula:-38.52880830509423, Newton-38.52876855954928

formula:-39.03328338327251, Newton-39.03321723892222

formula:-39.53451930072302, Newton-39.530271974818724

formula:-40.03259768075417, Newton-40.03265816598592

formula:-40.527596613889706, Newton-40.528105061782995

formula:-41.01959087233248, Newton-41.01865444304092

formula:-41.50865210780525, Newton-41.50864542427973

formula:-41.99484903432642, Newton-41.99515156726319

formula:-42.47824759730838, Newton-42.47204007750903

formula:-42.95891113021655, Newton-42.95812445534179

formula:-43.43690049989686, Newton-43.43674425749356

formula:-43.91227424156369, Newton-43.91176054168081

formula:-44.38508868433938, Newton-44.385147718498615

formula:-44.855398068145824, Newton-44.85662541940641

formula:-45.32325465267042, Newton-45.323208175649995

formula:-45.7887088190573, Newton-45.787584044329265

formula:-46.25180916491253, Newton-46.25181628582517

formula:-46.712602593156504, Newton-46.711876985175785

formula:-47.171134395206316, Newton-47.16497142215651

formula:-47.627448328927386, Newton-47.626430464531204

formula:-48.081586691753245, Newton-48.0841816076667

formula:-48.533590389336794, Newton-48.53339953718939

formula:-48.98349900006457, Newton-48.98387720863135

formula:-49.431350835736765, Newton-49.43367671978526

formula:-49.8771829986894, Newton-49.87730305502384

formula:-50.321031435612205, Newton-50.325927881950896

formula:-50.762930988294286, Newton-50.76211512557342

formula:-51.20291544151055, Newton-51.20290157226746

formula:-51.64101756824489, Newton-51.6404646361387

formula:-52.07726917242964, Newton-52.07748127667501

formula:-52.51170112936764, Newton-52.51167718473279

formula:-52.944343423989295, Newton-52.944302302445685

formula:-53.37522518708567, Newton-53.37476947951387

formula:-53.804374729647854, Newton-53.80442030323592

formula:-54.23181957543306, Newton-54.233359293810786

formula:-54.6575864918687, Newton-54.653590300965604

formula:-55.08170151939747, Newton-55.087360086668184

formula:-55.50418999935961, Newton-55.5043153050389

formula:-55.92507660050054, Newton-55.92337070110866

formula:-56.34438534418669, Newton-56.34315297622848

formula:-56.76213962840595, Newton-56.762006095065956

formula:-57.17836225062417, Newton-57.182732356547355

formula:-57.59307542956406, Newton-57.59837493900652

formula:-58.0063008259683, Newton-58.00628189567436

formula:-58.418059562404494, Newton-58.41838599281544

formula:-58.82837224216611, Newton-59.23894915189822

formula:-59.23725896731926, Newton-59.65003046677569

formula:-59.644739355942576, Newton-61.26094070101069

formula:-60.05083255860417, Newton-62.061111211465146

formula:-60.45555727411667, Newton-95.72251838797646

\subsection{Ničle Bi(x)}
formula:3.155519766331502, Newton-1.17371

formula:-3.2710787090748767, Newton-3.2712713561503115

formula:-4.830737748256738, Newton-4.831450021813814

formula:-6.169852124795421, Newton-6.169843786538271

formula:-7.376762079059813, Newton-7.376616165235318

formula:-8.491948846464942, Newton-8.49561668382789

formula:-9.538194379337316, Newton-9.545572198204308

formula:-10.529913506703094, Newton-10.529381998952783

formula:-11.476953551278095, Newton-11.478095059488933

formula:-12.3864171385825, Newton-12.384585929379117

formula:-13.263639522941714, Newton-13.263880804209784

formula:-14.11275680906862, Newton-14.112905890694384

formula:-14.937057412154147, Newton-14.937887400322495

formula:-15.739210351190474, Newton-15.734273300470338

formula:-16.52141955063437, Newton-16.521446412145387

formula:-17.28553162458124, Newton-17.285535660594405

formula:-18.033113287224996, Newton-18.02451712105774

formula:-18.765508284480074, Newton-18.764086737243684

formula:-19.48388013298923, Newton-19.478621616246272

formula:-20.1892447853962, Newton-20.186528384177446

formula:-20.88249599419317, Newton-20.875173269557386

formula:-21.56442528471297, Newton-21.561597666960026

formula:-22.23573788180338, Newton-22.23587773597104

formula:-22.897065554219793, Newton-22.905005896423408

formula:-23.548977079642437, Newton-23.547298619748908

formula:-24.191986850648995, Newton-24.19555893485858

formula:-24.826562012152888, Newton-24.826566266161016

formula:-25.453128427085126, Newton-25.449708505670852

formula:-26.07207569846679, Newton-26.072058818653325

formula:-26.683761425120984, Newton-26.684576308194718

formula:-27.2885148300763, Newton-27.294391906970688

formula:-27.886639871735955, Newton-27.886686622801964

formula:-28.47841792567866, Newton-28.478428133964595

formula:-29.064110107777644, Newton-29.064162296231512

formula:-29.643959295918396, Newton-29.644407721917197

formula:-30.218191897047273, Newton-30.217509425015898

formula:-30.78701939792176, Newton-30.78703050127969

formula:-31.35063973125558, Newton-31.3491594435392

formula:-31.909238483584566, Newton-31.908795587667797

formula:-32.46298996683685, Newton-32.468096595325264

formula:-33.01205817205683, Newton-33.01100565053907

formula:-33.55659762084005, Newton-33.55444518855837

formula:-34.09675412765602, Newton-34.09653776699074

formula:-34.63266548426774, Newton-34.63150882272717

formula:-35.16446207582101, Newton-35.16445239627691

formula:-35.6922674368108, Newton-35.69208554336992

formula:-36.21619875398748, Newton-36.21559956918772

formula:-36.7363673223012, Newton-36.73637557541924

formula:-37.25287895916828, Newton-37.25286523983991

formula:-37.765834381651786, Newton-37.767156266787396

formula:-38.275329550560045, Newton-38.2771735321796

formula:-38.78145598496326, Newton-38.78135679952282

formula:-39.28430105019801, Newton-39.27974724418718

formula:-39.78394822205711, Newton-39.78671992967341

formula:-40.280477329543686, Newton-40.28737484648495

formula:-40.77396477829067, Newton-40.773821889662685

formula:-41.26448375650675, Newton-41.266385900312386

formula:-41.75210442510105, Newton-41.75107737908031

formula:-42.23689409345656, Newton-42.237103464721486

formula:-42.71891738216253, Newton-42.714036076114844

formula:-43.19823637387693, Newton-43.197430819505755

formula:-43.674910753366724, Newton-43.67482027943288

formula:-44.14899793766616, Newton-44.1533014731412

formula:-44.62055319719727, Newton-44.620509257877735

formula:-45.089629768613115, Newton-45.08918045541097

formula:-45.55627896004906, Newton-45.55626486764024

formula:-46.02055024940101, Newton-46.020733795796055

formula:-46.48249137619077, Newton-46.482230175206986

formula:-46.942148427526014, Newton-46.94069837256733

formula:-47.39956591861494, Newton-47.39965397762286

formula:-47.854786868254514, Newton-47.84967431241946

formula:-48.307852869672466, Newton-48.30420082245228

formula:-48.75880415707066, Newton-48.75618369547595

formula:-49.207679668186024, Newton-49.2030570276277

formula:-49.6545171031586, Newton-49.65450039990594

formula:-50.099352979971236, Newton-50.0944234263575

formula:-50.54222268670363, Newton-50.548202875823904

formula:-50.98316053082284, Newton-50.9832409630268

formula:-51.422199785714675, Newton-51.42243472802071

formula:-51.85937273464332, Newton-51.859336326103396

formula:-52.29471071231238, Newton-52.29600489852289

formula:-52.72824414418604, Newton-52.728514795255236

formula:-53.16000258371715, Newton-53.15995867199868

formula:-53.590014747617914, Newton-53.589955419564255

formula:-54.01830854929814, Newton-54.01851716653977

formula:-54.44491113058688, Newton-54.444903894681254

formula:-54.8698488918446, Newton-54.86994827605265

formula:-55.29314752056544, Newton-55.29303225577967

formula:-55.714832018561395, Newton-55.71486541721097

formula:-56.134926727814054, Newton-56.13493531207754

formula:-56.55345535507366, Newton-56.55342910486751

formula:-56.97044099527886, Newton-56.967325056215614

formula:-57.38590615386648, Newton-57.384822440795034

formula:-57.79987276803498, Newton-57.795026617054454

formula:-58.21236222702162, Newton-58.2123702614951

formula:-58.623395391448845, Newton-58.62319972623049

formula:-59.032992611792075, Newton-59.03746010556881

formula:-59.441173746017405, Newton-59.441167462466325

formula:-59.84795817643466, Newton-59.84297210005587

formula:-60.253364825808355, Newton-446.5174841712581

\subsection{Komentar}
Izgleda torej, da sem ničle z Newtnovo metodo zadel, le na koncu se zgodi nekaj čudnega. Verjetno so moji parametri pri Newtnovi metodi neprimerni in je zaznal, kakšno približanje ničli kot samo ničlo. Tista zadnja "ničla" -95 pa je najbrž konvergiranje k neki čudni vrednosti(Opazil sem, da pogosto nekatere začetne točke konvergirajo kar nekam drugam, npr. kakšna zelo negativna točka konvergira na pozitivno os).

 Pri Bi(x) so ničle večinoma bolj uspele(Le zadnja po Newtnovi je spet izjemno čudna). Zanimiv rezultat pa je prva ničla po formuli za ničle pri Bi(x), prva je namreč pozitivna, kar vemo da ni. 

za Bi(x) mi pač za nek x okoli -60 newtnova metoda očitno konvergira k neki zelo oddaljeni ničli. Tega sem se rešil tako, da sem raje poiskal prvih 101 ničel, in potem odrezal najmanjšo(v tem primeru -446). Zdaj je rezultat tole:

formula:3.155519766331502, Newton-1.17371

formula:-3.2710787090748767, Newton-3.2712713561503115

formula:-4.830737748256738, Newton-4.831450021813814

formula:-6.169852124795421, Newton-6.169843786538271

formula:-7.376762079059813, Newton-7.376616165235318

formula:-8.491948846464942, Newton-8.49561668382789

formula:-9.538194379337316, Newton-9.545572198204308

formula:-10.529913506703094, Newton-10.529381998952783

formula:-11.476953551278095, Newton-11.478095059488933

formula:-12.3864171385825, Newton-12.384585929379117

formula:-13.263639522941714, Newton-13.263880804209784

formula:-14.11275680906862, Newton-14.112905890694384

formula:-14.937057412154147, Newton-14.937887400322495

formula:-15.739210351190474, Newton-15.734273300470338

formula:-16.52141955063437, Newton-16.521446412145387

formula:-17.28553162458124, Newton-17.285535660594405

formula:-18.033113287224996, Newton-18.02451712105774

formula:-18.765508284480074, Newton-18.764086737243684

formula:-19.48388013298923, Newton-19.478621616246272

formula:-20.1892447853962, Newton-20.186528384177446

formula:-20.88249599419317, Newton-20.875173269557386

formula:-21.56442528471297, Newton-21.561597666960026

formula:-22.23573788180338, Newton-22.23587773597104

formula:-22.897065554219793, Newton-22.905005896423408

formula:-23.548977079642437, Newton-23.547298619748908

formula:-24.191986850648995, Newton-24.19555893485858

formula:-24.826562012152888, Newton-24.826566266161016

formula:-25.453128427085126, Newton-25.449708505670852

formula:-26.07207569846679, Newton-26.072058818653325

formula:-26.683761425120984, Newton-26.684576308194718

formula:-27.2885148300763, Newton-27.294391906970688

formula:-27.886639871735955, Newton-27.886686622801964

formula:-28.47841792567866, Newton-28.478428133964595

formula:-29.064110107777644, Newton-29.064162296231512

formula:-29.643959295918396, Newton-29.644407721917197

formula:-30.218191897047273, Newton-30.217509425015898

formula:-30.78701939792176, Newton-30.78703050127969

formula:-31.35063973125558, Newton-31.3491594435392

formula:-31.909238483584566, Newton-31.908795587667797

formula:-32.46298996683685, Newton-32.468096595325264

formula:-33.01205817205683, Newton-33.01100565053907

formula:-33.55659762084005, Newton-33.55444518855837

formula:-34.09675412765602, Newton-34.09653776699074

formula:-34.63266548426774, Newton-34.63150882272717

formula:-35.16446207582101, Newton-35.16445239627691

formula:-35.6922674368108, Newton-35.69208554336992

formula:-36.21619875398748, Newton-36.21559956918772

formula:-36.7363673223012, Newton-36.73637557541924

formula:-37.25287895916828, Newton-37.25286523983991

formula:-37.765834381651786, Newton-37.767156266787396

formula:-38.275329550560045, Newton-38.2771735321796

formula:-38.78145598496326, Newton-38.78135679952282

formula:-39.28430105019801, Newton-39.27974724418718

formula:-39.78394822205711, Newton-39.78671992967341

formula:-40.280477329543686, Newton-40.28737484648495

formula:-40.77396477829067, Newton-40.773821889662685

formula:-41.26448375650675, Newton-41.266385900312386

formula:-41.75210442510105, Newton-41.75107737908031

formula:-42.23689409345656, Newton-42.237103464721486

formula:-42.71891738216253, Newton-42.714036076114844

formula:-43.19823637387693, Newton-43.197430819505755

formula:-43.674910753366724, Newton-43.67482027943288

formula:-44.14899793766616, Newton-44.1533014731412

formula:-44.62055319719727, Newton-44.620509257877735

formula:-45.089629768613115, Newton-45.08918045541097

formula:-45.55627896004906, Newton-45.55626486764024

formula:-46.02055024940101, Newton-46.020733795796055

formula:-46.48249137619077, Newton-46.482230175206986

formula:-46.942148427526014, Newton-46.94069837256733

formula:-47.39956591861494, Newton-47.39965397762286

formula:-47.854786868254514, Newton-47.84967431241946

formula:-48.307852869672466, Newton-48.30420082245228

formula:-48.75880415707066, Newton-48.75618369547595

formula:-49.207679668186024, Newton-49.2030570276277

formula:-49.6545171031586, Newton-49.65450039990594

formula:-50.099352979971236, Newton-50.0944234263575

formula:-50.54222268670363, Newton-50.548202875823904

formula:-50.98316053082284, Newton-50.9832409630268

formula:-51.422199785714675, Newton-51.42243472802071

formula:-51.85937273464332, Newton-51.859336326103396

formula:-52.29471071231238, Newton-52.29600489852289

formula:-52.72824414418604, Newton-52.728514795255236

formula:-53.16000258371715, Newton-53.15995867199868

formula:-53.590014747617914, Newton-53.589955419564255

formula:-54.01830854929814, Newton-54.01851716653977

formula:-54.44491113058688, Newton-54.444903894681254

formula:-54.8698488918446, Newton-54.86994827605265

formula:-55.29314752056544, Newton-55.29303225577967

formula:-55.714832018561395, Newton-55.71486541721097

formula:-56.134926727814054, Newton-56.13493531207754

formula:-56.55345535507366, Newton-56.55342910486751

formula:-56.97044099527886, Newton-56.967325056215614

formula:-57.38590615386648, Newton-57.384822440795034

formula:-57.79987276803498, Newton-57.795026617054454

formula:-58.21236222702162, Newton-58.2123702614951

formula:-58.623395391448845, Newton-58.62319972623049

formula:-59.032992611792075, Newton-59.03746010556881

formula:-59.441173746017405, Newton-59.441167462466325

formula:-59.84795817643466, Newton-59.84297210005587

formula:-60.253364825808355, Newton-60.2534012885078



Kar izgleda veliko bolj v redu.
\end{document}


